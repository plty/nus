\documentclass[12pt,letterpaper,pdftex]{report}
\usepackage{graphicx}
\usepackage{amsmath}
\usepackage{amssymb}
\usepackage{amsfonts}
\usepackage{amsthm}
\usepackage{bm}
\usepackage{paralist}
\usepackage{booktabs}
\usepackage{url}
\usepackage{xspace}
\usepackage[pdftex,letterpaper,left=1in,top=1in,bottom=1in,right=1in,includeheadfoot]{geometry}
\usepackage{fancyhdr}
\usepackage[shortlabels]{enumitem}
\pagestyle{fancy}
\lhead{\footnotesize Tutorial 1: CS3243 Introduction to AI \\ \bf {Issued: Aug 19, 2019}}
\rhead{\footnotesize Semester 1, AY 2019/20 \\ \bf{Due: Week 3 Tutorial}}
\lfoot{}
\cfoot{}
\rfoot{\thepage}
\renewcommand{\headrulewidth}{1pt}
\usepackage{times}
\usepackage[pdftex]{epsfig}

\usepackage{algorithm}
\usepackage{algcompatible}

\usepackage{comment}
\specialcomment{solution}{\paragraph{\textbf{Solution:}}}{\par}



\usepackage{tabularx}
\newtheorem{theorem}{Theorem}
\newtheorem{proposition}[theorem]{Proposition}
\newtheorem{lemma}[theorem]{Lemma}
\newtheorem{corollary}[theorem]{Corollary}
\theoremstyle{definition}
\newtheorem{definition}[theorem]{Definition}

\newcommand{\ma}[1]{{\bf #1}}
\newcommand{\ve}[1]{{\mathbf #1}}
\newcommand{\set}[1]{{\mathcal #1}}
\newcommand{\E}{\mathbb{E}}
\newcommand{\defeq}[0]{\ensuremath{\stackrel{\triangle}{=}}}
\renewcommand{\cal}[1]{\mathcal{#1}}
\newcommand{\tup}[1]{\langle #1 \rangle}
\newcommand{\true}{\texttt{true}\xspace}
\newcommand{\false}{\texttt{false}\xspace}
\newcommand{\Neighbors}{\mathtt{Neighbors}}
\begin{document}

{\bf Instructions}:
\emph{
	\begin{compactitem}
	\item Your solutions for this tutorial must be TYPE-WRITTEN.
	\item Submit your solution(s) to your tutor. You can make another copy for
	yourself if necessary. Late submission will NOT be entertained.
	\item YOUR SOLUTION TO QUESTION $1$ will be GRADED for this tutorial.
	\item You can work in pairs, but each of you should submit the solution(s) individually.
	\item Include the name of your collaborator in your submission.
	\end{compactitem}
}

\vspace{5mm}

\begin{table}[]
	\begin{tabular}{|l|p{5cm}|l|p{2cm}|}
	\hline
	Name         &  & Matric number                &  \\ \hline
	Collaborator &  & Collaborator's matric number &  \\ \hline
	\end{tabular}
	\end{table}

\begin{enumerate}


\item You are tasked with the creation of an interactive agent that would be
deployed in an automatic coffee-vending machine. The agent needs to work as a
chatbot to converse with the user. Answer the questions below which would be
crucial in the design of your agent:

	\begin{compactenum}
		\item How would you define the agent’s state and action space?
		\item What would the utility function of the agent involve?
		\item What kind of exploration actions could the system take?
		\item For a highly flexible chatbot deployed as the agent, what kind of
		action should it refrain from taking?
	\end{compactenum}

\begin{solution}
	User agent's state: [Credit, Chosen Coffee, Machine State]
\end{solution}


\item You have been appointed as the lead engineer in the development team of
NUSmart Shuttle Bus - the new autonomous shuttle in
NUS~\footnote{\url{https://uci.nus.edu.sg/oca/latest-news/nusmart-shuttle-a-fully-autonomous-vehicle/}}.
Define the characteristics of the task environment in which the shuttle has to
be deployed. Support your answer with sufficient reasoning behind your choices.
	
	\begin{compactenum}
		\item Comment on the observability aspect of the environment. 
		\item Is the environment a single-agents, collaborative multi-agent or
		competitive multi-agent environment? Justify your answer.
		\item Will the environment be deterministic or stochastic? Give an
		example scenario to support your answer.
		\item Would you model the environment to be episodic or sequential?
		Explain your choice.
	\end{compactenum}

	
\begin{figure}[ht!]
	\centering
	\includegraphics[scale=0.1]{images/autonomous}
	\caption{NUSmart autonomous shuttle.}
	\label{autonomous}
\end{figure}

\begin{solution}
	\textbf{Your solution here}
\end{solution}


\item Are reflex actions (such as flinching your hand from a hot stove)
rational? Are reflex actions intelligent?

\begin{solution}
	\textbf{Your solution here}
\end{solution}


\item Weizenbaum's ELIZA program simulates the behavior of a
psychotherapist carrying out a conversation with a patient. It
basically works by finding keywords in the user's input so as to fire
certain rules based on the keywords. Which AI definition does ELIZA
fit in? (Thinking humanly? Acting humanly? Thinking rationally? Acting
rationally?) Discuss how an ELIZA-like system will behave, if it is
modeled according to each of the four agent types, namely, ``simple
reflex agent'', ``model-based reflex agent'', ``goal-based agent'',
and ``utility-based agent''.

\begin{solution}
	\textbf{Your solution here}
\end{solution}


\end{enumerate}



\end{document}
