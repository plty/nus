\documentclass[12pt,a4paper,pdftex]{report}
\usepackage{graphicx}
\usepackage{amsmath}
\usepackage{amssymb}
\usepackage{amsfonts}
\usepackage{amsthm}
\usepackage{bm}
\usepackage{paralist}
% \usepackage{sudoku}
\usepackage{booktabs}
\usepackage{url}
\usepackage{xspace}
\usepackage[pdftex,letterpaper,left=1in,top=1in,bottom=1in,right=1in,includeheadfoot]{geometry}
\usepackage{fancyhdr}
\usepackage[shortlabels]{enumitem}
\pagestyle{fancy}
\lhead{\footnotesize Tutorial 2: CS3243 Introduction to AI \\ \bf {Issued: Aug 26, 2019}}
\rhead{\footnotesize Semester 1, AY 2019/20 \\ \bf{Due: Week 4 Tutorial}}
\lfoot{}
\cfoot{}
\rfoot{\thepage}
\renewcommand{\headrulewidth}{1pt}
\usepackage{times}
%\usepackage[boxed]{algorithm}
%\usepackage[noend]{algorithmic}
\usepackage[pdftex]{epsfig}

\usepackage{algorithm}
\usepackage{algcompatible}

\usepackage{comment}
\specialcomment{solution}{\paragraph{\textbf{Solution:}}}{\par}
%\excludecomment{solution}



\usepackage{tabularx}
\newtheorem{theorem}{Theorem}
\newtheorem{proposition}[theorem]{Proposition}
\newtheorem{lemma}[theorem]{Lemma}
\newtheorem{corollary}[theorem]{Corollary}
\theoremstyle{definition}
\newtheorem{definition}[theorem]{Definition}

\newcommand{\ma}[1]{{\bf #1}}
\newcommand{\ve}[1]{{\mathbf #1}}
\newcommand{\set}[1]{{\mathcal #1}}
\newcommand{\E}{\mathbb{E}}
% The ``defined as'' symbol.
\newcommand{\defeq}[0]{\ensuremath{\stackrel{\triangle}{=}}}
\renewcommand{\cal}[1]{\mathcal{#1}}
\newcommand{\tup}[1]{\langle #1 \rangle}
\newcommand{\true}{\texttt{true}\xspace}
\newcommand{\false}{\texttt{false}\xspace}
\newcommand{\Neighbors}{\mathtt{Neighbors}}
\begin{document}


{\bf Instructions}:
\emph{
	\begin{compactitem}
	\item Your solutions for this tutorial must be TYPE-WRITTEN.
	\item Print and submit your solution(s) to your tutor. You can make another
	copy for yourself if necessary. Late submission will NOT be entertained.
	\item YOUR SOLUTION TO QUESTION $1$ will be GRADED for this tutorial.
	\item You can work in pairs, but each of you should submit the solution(s) individually.
	\item Include the name of your collaborator in your submission.
	\end{compactitem}
}

\vspace{5mm}
% \renewcommand*\sudokuformat[1]{\Large\rmfamily#1}
% \setlength\sudokusize{8cm}

\newcommand{\skipit}[1]{}


\vspace{5mm}

\begin{table}[]
	\begin{tabular}{|l|p{5cm}|l|p{2cm}|}
	\hline
	Name         &  Jerrell Ezralemuel & Matric number                &  A0181002B\\ \hline
	Collaborator &  & Collaborator's matric number &  \\ \hline
	\end{tabular}
	\end{table}




\begin{enumerate}

	\item The Towers of Hanoi is a famous problem for studying recursion in computer
	science and recurrence equations in discrete mathematics. We start with \(N\)
	discs of varying sizes on a peg (stacked in order according to size), and two
	empty pegs. We can move a disc from one peg to another, but we are never allowed
	to move a larger disc on top of a smaller disc. The goal is to move all the
	discs to the rightmost peg

    In this problem, we will formulate the Towers of Hanoi as a search problem.  

	\begin{compactenum}
	\item Propose a state representation for the problem.
	\item What is the start state?
	\item From a given state, what actions are legal?
	\item What is the goal test?
	\end{compactenum}

	\begin{solution}
            \begin{enumerate}
                \item State is represented by an array $A$ of size $n$ in which $A_i$ represents 
                    in which peg is the $(i+1)$-th smallest disc is located.
                \item $[\underbrace{0, 0, \dots, 0, 0}_n]$
                \item Moving smallest disc in a peg to another peg which the smallest disc in it is larger than the current disc
                \item $[\underbrace{2, 2, \dots, 2, 2}_n]$
            \end{enumerate}
	\end{solution}
\end{enumerate}

\end{document}
